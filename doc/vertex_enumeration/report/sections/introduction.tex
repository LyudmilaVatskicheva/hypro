\section{Introduction}
Computer scientists are used to deal with discrete systems, i.e. systems whose evolution can be described by a sequence of discrete state changes. A widespread example is the program analysis: during the execution of a program, the system is abstracted into a machine which states changes every time an instruction is executed.
Today, computer science is expected to be able to ensure the safety of very complex systems, such as planes or nuclear reactors, which state are described non only by a controller, which is discrete, but also by physical variables, which evolve continuously over the time. Such systems, with a continuous and a discrete component are called hybrid systems. These systems are omnipresent, whenever a computer interacts with continuous quantities there is an hybrid system, even for thing as simple as a thermostat with the temperature.


To study these systems, they are abstracted into hybrid automata. These automata are then model checked. As an example let's take a thermostat, which correspond to an hybrid system, and turn it into an hybrid automata.

An hybrid system is described by a set of possible discrete states and a set of continuous variables. The evolution of the variables is determined by the current discrete state. The current state of the system is defined by the discrete state it is in and by the current value of the continuous variables. A transition between two discrete states can be done under certain conditions (called a guard) over the variables, a well as staying in a given state (called the invariant of the state). A transition can reassign the variables.  

\begin{example} heat controller
\end{example}

\begin{table}
\centering
\begin{tabular}{| c | c | c | c | c |}
	\hline	
	Sub- & derivative & conditions & bounded  & unbounded \\
	classes & & & reachability & reachability \\ \hline
	TA & $\dot x=1$ & $x\overset{?}{=}c$ &\checkmark &\checkmark \\ \hline
	& & $x\overset{?}{\in} [c_1;c_2]$ & &   \\	
   	IRA & $\dot x\in [c_1;c_2]$ & jump must reset &\checkmark &\checkmark \\ 
   	& & $x$ when $\dot x$ changes & &\\ \hline
   	RA & $\dot x\in [c_1;c_2]$ & $x\overset{?}{\in} [c_1;c_2]$ &\checkmark &X \\ \hline
   	LHA I & $\dot x=c$ & $x\overset{?}{=}g_{linear}$ &\checkmark &X \\ \hline
   	LHA II & $\dot x=f_{linear}$ & $x\overset{?}{=}g_{linear}$ &X &X \\ \hline
   	HA  & $\dot x=f$ & $\dot x=g$ &X &X \\ \hline
\end{tabular}
\label{tab_complexity}
\caption{Decidability results for subclasses of automata. $c$, $c_1$, $c_2$ are constants, $f$ and $g$ are function of other variables. TA~=~timed automata, IRA~=~ initialised rectangular automata, RA~=~rectangular automata, LHA I~=~hybrid automata with constant derivatives, LHA~II~=~hybrid automata with linear ODE's, HA~=~general hybrid automata.}
\end{table}

Analyse such systems is very difficult, the table~\ref{tab_complexity} gives the complexity of model checking different classes of automata. Several methods exist to deal with this problem, theorem proving (model for a solver), interval based methods(model it for an SMT solver) and flowpipe computation. This paper contributes to the flowpipe-construction-based reachability analysis methods. The idea is to discretize the time and, at each time step, determine set containing all the possible values for the continuous variables for that time step. The successive representations of the set define a flowpipe=> image.

There exists different sub-methods for the flowpipe construction, each corresponding to different methods to describe the set where of the possible values of the variables. Zonotopes, support functions and boxes are examples of representation. A lot of operations have to performed on these sets along the way, for instance intersections and unions. This paper proposes a method to switch between two representations of polyhedra: the $V$ representation and the $H$ representation according to the theorem~\ref{thm_representation}. First some notations and definitions for the rest of the paper:
 
\begin{definition}[Some geometry]
	The problem is studied in $\mathbb{R}^d$.
	\begin{itemize}
	\item $conv(V)$ defines the convex hull of the set of vertices $V$: $\{ x\in\mathbb{R}^d| x=\sum_{v\in V} \lambda_v v, \sum_{v\in V} \lambda_v =1, \forall v \in V, \ 0\leq \lambda_v \in \mathbb{R} \}$.
	\item $cone(C)$ defines the conic hull of the vectors in $C$: $\{ x\in\mathbb{R}^d| x=\sum_{c\in C} \lambda_c c, \forall c \in C,\ 0\leq \lambda_c \in \mathbb{R} \}$.
	\item $lineal(L)$ is the linear space generated by $L$: $\{ x\in\mathbb{R}^d| x=\sum_{l\in L} \lambda_l l, \forall l \in L,\ \lambda_c \in \mathbb{R} \}$. 
	\item In the paper, a sum between two sets is the Minkowsky sum: $S_1+S_2=\{s_1+s_2|s_1\in S_1,\ s_2 \in S_2 \}$.
	\item An half-space $h$ is an area of $\mathbb{R}^d$ defined by a normal vector $a$ and a constant $b$, $h=\{x\in\mathbb{R}^d|x.a\leq b\}$, its border is the hyperplane $\{x\in\mathbb{R}^d|x.a = b\}$. A family of half-spaces are said independent is their normal vectors are independent, and independent hyperplanes are defined respectively , $d$ independent hyperplanes intersect in a vertex.
	\item Let $A$ be a matrix which rows are normal vectors of a finite set of half-spaces and $B$ the column vector composed by the corresponding constants. $P(A,B)=\{x\in\mathbb{R}^d|Ax\leq B\}$ is the set of the points contained in all the half-spaces.
	\item An $H$-polyhedron $P$ is the intersection of a given finite set of half-spaces. There exist $A$ and $B$ such that $P=P(A,B)$.
	\item A $V$-polyhedron $P$ is the sum of the convex hull of a finite set of points and a conic hull of a finite set of points, $P=conv(V)+cone(C)$ for some finite $V$ and $C$.
	\item A polyhedron is an $H$-polyhedron or a $V$-polyhedron.
	\item A polytope is a bounded polyhedron (and respectively with $H$ and $V$ polytopes).
	\end{itemize}
\end{definition}


\begin{theorem}[Polyhedra representation]
A Subset $P\subseteq\mathbb{R}^d$ is an $H$-polyhedron if and only if it is a $V$-polyhedron.
\label{thm_representation}
\end{theorem} 
These two representations have their own advantages and disadvantages, these are summed up in the following tableau:

\begin{tabular}{| c || c | c | c | c |}
	\hline	
				    & & & & \\ \hline
	$V$-Polyhedra   & & & & \\ \hline
   	$H$-Polyhedra   & & & & \\ \hline
\end{tabular}

Thus, being able to switch between these two representations is crucial.

The hosting research group develops in the context of the HyPro project an open-source stand-alone C++ library for the most relevant geometric state set representations
%, covering boxes, polytopes, ?orthogonal polytopes, zonotopes, support functions and ?Taylor models. 
Furthermore the library allows the combination of different representations and over-approximative conversion between them. So far it includes







=> state representation
polyhedra theory
advantage/inconveniants of both representations








Some operations might have to be applied to the polyhedron such like intersections, unions ect. The difficulty of these opérations greatly depends on the description used for the polyhedron.

Def: V poly
Def: H poly
Theorem: there is a V and a H descripion for each polyhedron.

Being able to swich between these descripions is an important issue to overcome in reachability analysis.

The following part describe an algorithm to convert a bounded H polyhedron (an H polytope) into its V description and extend it to the unbounded case.